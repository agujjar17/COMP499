\chapter{Introduction}
\label{chap1}
\section{Background}
Breast cancer is the most commonly diagnosed cancer when it comes to cancer deaths, and in women, second only to lung cancer \cite{CanadianCancer2004}. It costs about one in three women diagnosed with cancer each year \cite{CancerJournal}. There are approximately 5500 people and 28,600 new deaths from breast cancer by 2022 in Canada. It is estimated that one in eight women will develop breast cancer in her lifetime, and one in thirty-four will die of disease \cite{PublicHealthAgencyofCanada2024}.
\newline Breast cancer risk factors include age; 70\% of newly diagnosed cases involve someone 50 years of age or older \cite{CanadianCancer2004}. Thus, in keeping with similar recommendations in other Western nations, Health Canada recommends screening mammography for women over 50 every two years. X-ray mammography is considered the gold standard for early cancer diagnosis because it can detect subtle or minute cancer symptoms that are overlooked by self-examination or routine medical exams \cite{CanadianCancer2004}.
\newline In addition, the difficulties with present breast cancer detection techniques, such as ultrasound and mammography, call for a critical analysis of their shortcomings, which emphasizes the importance and urgency of developing new diagnostic tools. For the European guidelines on quality assurance in breast cancer diagnoses, there has been a need to review and strive for better screening and diagnosing techniques. Raising the detection techniques is an unavoidable need, so the patient's anxiety level can be reduced, and meaningless tests and false positive rates decrease, leading to improved lives and constant success of screening programs for breast cancer. \cite{QualityMeasure}. This emphasizes how crucial it is to improve breast cancer detection techniques to lower the number of unwarranted recalls, false positives, and related patient distress.
\section{Rationale for Research}
For breast cancer detection, our initial goal is to narrow down the existing gaps and form a reliable data-transforming algorithm using Digital Imaging and Communications in Medicine (DICOM) files which can be achieved with the help of wavelet functions. Unlike claiming a new model, we utilize the existing technique by Michel Bernett \cite{Michael2006} and apply it with different datasets to assist in better early breast cancer detection.
\newline Our approach is implemented in two steps, the first step is to apply wavelet transformations to DICOM files, and the second is to use these wavelet features and extract statistical parameters from these images. This procedure will help us parse out important statistical features from the images. To sum up, in the data model that has been created, the retrieved statistical information is assigned as indispensable inputs to a simple machine learning model. The core purpose of this model is to find out with high accuracy if the patient has breast cancer or not. 
\newline We are going to evaluate the quality of our proposed methods by comparing its performance to existing techniques, aiming to demonstrate its capability to provide additional discriminative features for accurate predictions. Besides, we are also researching to investigate the discriminative features of DICOM images after wavelet transformation in connection with the context of breast cancer detection. 
\newline Moreover, data validation will not be the only competitive advantage of our approach. We would also like to find out the sensibility of the statistical attributes of wavelet-transformed DICOM images. The purpose of this is to find out the information these features can provide in this complex field of breast cancer detection. 
\section{The Significance of Wavelet-Based Approaches}
Wavelet-based techniques are increasingly becoming the methods of choice in the field of research because they do a great job in several applications with one major use in statistical process monitoring being the fact that they are very effective \cite{WaveletStatistics}. Wavelet analysis is proven to be an effective tool because of its ability for data processing involving different resolutions of scales. This helps solve problems like noise processing, autocorrelation, and the treatment of anomalous data \cite{WaveletStatistics}. The wavelet's performance in multivariate techniques, especially in the area of multiscale statistical process monitoring, adds another dimension of flexibility and effectiveness as these complicated data analysis processes are handled \cite{WaveletStatistics}. Furthermore, researchers have proved that wavelet analysis also serves well as a pre-treatment method for data cleansing as well as boosting the accuracy of subsequent statistical models \cite{Michael2006}.
\newline Our research is aimed at discovering innovative methods based on wavelet techniques which will result in more accurate and fast detection of breast cancer, with applications of this achievement both in medicine and other fields. The approach we will use will consist of a combination of machine learning techniques, which will be enhanced by wavelet-based feature extraction. Our method intends to enhance the precision and capability of the model by applying wavelet transformations to DICOM images and getting statistical information about these changes.

\section{Objectives of the Study}
The most important part of our research is to prove that our suggested method is better than the ways people apply the methodology now. We therefore use the comparative technique which will show that the wavelet-based approach has an edge over others in terms of its effectiveness and power to distinguish those people who are most likely to have breast cancer.
\newline The goal of our study is precisely to understand the importance of these statistical parameters obtained from the wavelet-transformed DICOM images and then apply such information to the breast cancer identification methods. The information that we are trying to achieve through our research is vital for the diagnosis and treatment of breast cancer. Thus, our research will achieve this with the ultimate goal of changing the method of cancer diagnosis and addressing some of the existing gaps in the understanding of how to detect breast cancers which results in improving the outcomes of the treatment and also the whole process of medical care.