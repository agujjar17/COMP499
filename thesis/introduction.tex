\chapter{Introduction}
\label{chap1}
\section{Background}
In terms of cancer-related deaths, breast cancer is the most common cancer diagnosed in women and comes in second only to lung cancer \cite{CanadianCancer2004}. It affects around one in three women who receive a cancer diagnosis each year \cite{CancerJournal}. In Canada, 5,200 deaths and 21,200 new cases of breast cancer were recorded in 2004. This number exceeded the number of female lung cancer deaths and new diagnoses \cite{Michael2006}. 
\newline Breast cancer risk factors include age; 70\% of newly diagnosed cases involve someone 50 years of age or older \cite{CanadianCancer2004}. Thus, in keeping with similar recommendations in other Western nations, Health Canada recommends screening mammography for women over 50 every two years. X-ray mammography is considered the gold standard for early cancer diagnosis because it can detect subtle or minute cancer symptoms that are overlooked by self-examination or routine medical exams \cite{CanadianCancer2004}.
\newline In addition, the difficulties with present breast cancer detection techniques, such ultrasound and mammography, call for a critical analysis of their shortcomings, which emphasizes the importance and urgency of developing new diagnostic tools. Although quality assurance techniques like consensus and double reading work to reduce false alarms, some regions still have recall rates between 3 and 4\% \cite{QualityMeasure}. This emphasizes how crucial it is to improve breast cancer detection techniques in order to lower the number of unwarranted recalls, false positives, and related patient distress.
\section{Rationale for Research}
Given the challenges in breast cancer detection, our primary objective is to address existing gaps and develop a comprehensive method for extracting features from Digital Imaging and Communications in Medicine (DICOM) files using wavelet functions. Unlike claiming a new model, our focus is on implementing an existing method and testing it with new datasets to improve early breast cancer detection. 
\newline Our methodology unfolds by carefully applying wavelet transforms to DICOM pictures. We attempt to extract important statistical data from the converted photos using this procedure. In turn, a simple machine learning model uses these retrieved information as priceless inputs. This model's main objective is to identify, with high accuracy, if a particular patient is likely to have breast cancer. 
\newline We intend to assess the effectiveness of our method by comparing its performance to existing techniques, aiming to demonstrate its capability to provide additional discriminative features for accurate predictions. Additionally, our study seeks to explore the significance of the statistical characteristics obtained from wavelet-transformed DICOM images in the context of breast cancer detection.
\newline In addition to the practical validation of our approach, we hope to investigate the subtle meaning of the statistical features obtained from wavelet-transformed DICOM pictures. By doing this, we want to uncover the special perspectives that these traits might provide in the complex field of breast cancer detection.
\section{The Significance of Wavelet-Based Approaches}
Wavelet-based approaches have gained popularity in recent years due to their great efficacy in a variety of applications, particularly in the field of statistical process monitoring \cite{WaveletStatistics}. Wavelet analysis's intrinsic flexibility, which enables operations at different resolutions or scales, has been shown to be helpful in overcoming obstacles including measurement noise, autocorrelation, and the management of non-normal data \cite{WaveletStatistics}. Wavelets' performance in multivariate techniques, especially in the context of multiscale statistical process monitoring, highlights its adaptability and usefulness in the analysis of complicated data \cite{WaveletStatistics}. Additionally, research indicates that wavelet analysis is useful as a pre-processing technique, helping with data denoising and improving the performance of ensuing statistical models in addition to its ability to reduce noise \cite{Michael2006}.
\newline Our work seeks to maximize the potential of wavelet-based techniques to improve early breast cancer diagnosis, taking inspiration from these achievements in several scientific fields. Our novel strategy involves combining machine learning approaches with wavelet-based feature extraction. Our technique aims to improve the accuracy and discriminative capacity of prediction models for breast cancer diagnosis by applying wavelet modifications to DICOM pictures and deriving statistical information from these altered images. By expanding the use of wavelet applications into the crucial field of medical imaging, this study strengthens the basis already in place.
\section{Objectives of the Study}
A crucial component of our study entails a thorough performance comparison between our suggested methodology and current methods. This comparative analysis is necessary to show the effectiveness and discriminative power of our method in detecting possible cases of breast cancer. We predict that our approach will perform much better than present standards, creating a new benchmark for accuracy and dependability in breast cancer diagnosis, drawing comparisons with the success of wavelet-based methods in statistical process monitoring.
\newline Our research attempts to explore the subtle meaning of the statistical features obtained from wavelet-transformed DICOM pictures in the particular setting of breast cancer identification. This investigation is motivated by the knowledge that these traits, when thoroughly examined, can provide important new information that is essential for an early diagnosis and course of action. By fulfilling these goals, our research hopes to change the paradigm in diagnostic techniques and close important gaps in the existing understanding of breast cancer detection, which will have a significant impact on patient outcomes and medical procedures.