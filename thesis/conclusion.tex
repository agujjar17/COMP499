\chapter{Conclusion}
\label{chap5}
\section{Recapitulation of Key Findings}
The purpose of this study was to create a wavelet-based method for detecting breast cancer by using machine learning models on statistical parameters taken from DICOM pictures. One of the study's main conclusions is that machine learning algorithms, such as Decision Tree, Random Forest, Support Vector, and Logistic Regression combined with wavelet based statistical parameters are not successfully to diagnose breast cancer. Using an extensive dataset, the models were assessed and trained, resulting in a range of performance outcomes.

A number of statistical factors, such as mean, standard deviation, skewness, and kurtosis, were shown to be useful in describing subtle characteristics seen in breast tissue, which enhanced the models' ability to discriminate. Model comparisons showed how crucial it is to take into account various algorithmic techniques in order to achieve the best results.
\section{Implications for Breast Cancer Detection}
The results obtained in this study have significant implications for the field of breast cancer detection. The successful integration of machine learning models with wavelet-based feature extraction techniques showcases the potential of computational approaches in enhancing the accuracy and efficiency of breast cancer screening. The sensitivity and specificity analyses provide insights into the models' ability to identify positive and negative cases, informing the development of more robust and reliable diagnostic tools.

\section{Future Directions}
While this study has made strides in utilizing machine learning for breast cancer detection, there are avenues for future research. Firstly, the incorporation of more diverse datasets, including different demographics and imaging modalities, can enhance the generalizability of the models. Additionally, exploring advanced deep learning architectures and techniques may further improve the performance of breast cancer detection systems.

\section{Conclusion}
To sum up, this study adds to the continuing attempts to use computational techniques to detect breast cancer. Promising outcomes are shown when wavelet-based feature extraction is integrated with machine learning models. The study emphasizes the necessity of a thorough and sophisticated strategy that takes into account the advantages and disadvantages of various algorithms. The combination of clinical knowledge and computer-aided diagnostic techniques has enormous potential for enhancing breast cancer prognosis and early detection as technology develops.