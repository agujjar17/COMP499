\addcontentsline{toc}{chapter}{Abstract}
\begin{center}
	\textbf{\large Abstract}
\end{center}
\noindent
Breast cancer remains a significant global health concern, emphasizing the need for advanced and accurate detection methods to improve patient outcomes. This research introduces a novel approach to early breast cancer detection using digital imaging and wavelet-based statistical analysis. The study leverages DICOM files containing mammographic images, employing wavelet transformations to extract essential features and statistical characteristics from the images. These extracted features serve as input to a predictive model, facilitating the identification of patterns indicative of breast cancer.

By utilizing statistical measures derived from wavelet-transformed images, the model aims to provide a robust and interpretable framework for breast cancer prediction. The dataset comprises a diverse set of mammograms, ensuring the model's generalizability across different patient profiles. The research contributes to the field of early breast cancer detection by introducing a methodology that combines advanced image processing techniques with machine learning. The model's interpretability is enhanced through the extraction of statistical features, providing insights into the underlying patterns contributing to cancer detection. The ultimate goal is to develop a reliable and transparent tool that complements existing diagnostic practices, potentially improving the efficiency of breast cancer screening and contributing to better patient outcomes. 