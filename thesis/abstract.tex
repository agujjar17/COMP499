\addcontentsline{toc}{chapter}{Abstract}
\begin{center}
	\textbf{\large Abstract}
\end{center}
\noindent
Breast cancer is a major global health issue, that needs advanced and valid healthcare methods and diagnosis to increase the patient’s chances of survival. To allow researchers to facilitate in this space, we are going to conduct our research on early breast cancer detection using wavelet methods and the statistical composition of digital images. We employ DICOM-encapsulated mammography images and apply a wavelet modification to extract important features and statistical attributes from the pictures. By using these characteristics, a prediction algorithm can identify trends that point to the possibility of breast cancer.

The objective here is to create an understandable model for the diagnosis of breast cancer, which will be tested using different factors in many classes of classifiers using the statistics picked from the pictures of breast tissues. The study sample sets called the mammograms, detect specific patient attributes and aid in the examination of model accuracy across different clinical conditions and a wider range of patients. This novel method combines the advancement of image processing tools and machine learning with structured feature extraction to capture such breast tissue patterns that can be used to detect cancer. The model aims to be accurate, robust and straightforward, this is a step aimed at replacing current tools and techniques on breast cancer diagnosis, therefore early detection of breast cancer will be greatly facilitated.