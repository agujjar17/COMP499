\chapter{Methodology}
\label{chap3}
\section{Datasets}
In this part, the various datasets, which are used for training, testing, and validating the developed breast cancer detection models, are summarized. The employment of multiple datasets raises the resilience and portability of the proposed model, covering diverse demographics, imaging modalities, and annotation types.
\subsection{VinDr-Mammo Dataset}
The VinDr-Mammo dataset proposed by Pham et al. \cite{Vindir}, crucial for advancing breast cancer detection research, is structured within a project directory that comprises DICOM files, annotations, and metadata across 5,000 mammography exams. Specifically, the dataset organization includes:
\begin{itemize}
  \item images: A subfolder containing 5,000 subdirectories for each exam, with each subdirectory named after a hashed study identifier. Each contains four DICOM files representing two standard views (CC and MLO) of each breast.
  \item breast-level-annotations.csv: Provides BI-RADS assessment and image metadata for each image, including study-id, series-id, laterality (L or R), view-position (CC/MLO), image dimensions (height and width), breast-birads (BI-RADS assessment), breast-density, and the dataset split (training/test).
  \item finding-annotations.csv: Contains detailed annotations for breast abnormalities per image, including the metadata from the image and specific annotations like finding-categories, finding-birads (BI-RADS assessment of the finding), and bounding box coordinates (xmin, ymin, xmax, ymax).
  \item metadata.csv: Offers additional details relevant for research, such as the patient's age, and imaging device's model and manufacturer, provided by DICOM tags.
\end{itemize}
This dataset's relevance is not just because of its large size but the annotations which make it an excellent resource for building and validating breast cancer detection models especially in cases where interpretability is crucial.
\subsection{RSNA Screening Mammography Breast Cancer Detection Dataset}
The RSNA Screening Mammography Breast Cancer Detection dataset \cite{RSNA}, is integral to the study's exploration of breast cancer identification from screening exams. The dataset contains radiographic breast images, and the primary goal is to identify cases of breast cancer in mammograms. This dataset provides a detailed framework for breast cancer detection through screening mammography, incorporating DICOM images and extensive metadata. The organization allows for nuanced analysis and model training:

\begin{itemize}
  \item DICOM Images: Featuring mammograms, potentially in jpeg 2000 format, across roughly 8,000 patients. Each patient typically has four images.
  \item site-id: Identifier for the source hospital.
  \item patient-id: Unique code for the patient.
  \item image-id: Unique identifier for the image.
  \item laterality: Indicates if the image is of the left or right breast.
  \item view: Image orientation, with two views per breast being standard.
  \item age: Patient's age in years.
  \item implant: Indicates the presence of breast implants.
  \item density: Breast tissue density rating from A (least dense) 
  to D (most dense).
  \item machine-id: Identifier for the imaging device.
  \item cancer: Malignancy status of the breast, with follow-up details 
  like biopsy, invasive status, and BIRADS assessments provided for training data.

\end{itemize}
This dataset's significance lies in its real-world applicability, aligning with the challenges faced in breast cancer screening programs. The diverse set of parameters provides a holistic view, enabling the study to address nuanced aspects of breast cancer detection.
\subsection{INbreast Dataset}
The INbreast dataset, introduced by Moreira et al. \cite{Inbreast} in their technical report, is a valuable resource for breast cancer research and model development. This dataset contains 410 photos, that is 115 cases total in the INbreast archive which were used for the research in this essay. These ninety incidences involved women were with images (MLO and CC) of the two breasts being photographed that give us a total of four photographs for every individual case. These cases have 49 of them who had breast amputation, in which they are their only document of one of their breasts. DICOM format is the encoding used for the images in the collection that has characters regarding the information of the images, for instance, info related to equipment of capturing, size, voxel, color mode, quantity of bits, and so forth. The material on the planning of mammogram by the database will be tested on different type of lesions, for instance masses, calcification, asymmetries, and distortions. As well, the database includes the segmentations of defects which are associated to medical subspecialty, such as imbalance, multi-directional growth, calcification etc.

The INbreast dataset's strengths lie in its focus on full-field digital mammograms, wide variability of cases, and the provision of precise annotations, making it a valuable asset for researchers aiming to develop and evaluate computer-aided detection and diagnosis systems for breast cancer.
\section{Computational Environment}
Python 3.9 served as the kernel language for the high-level wavelet-based breast cancer detection system. Tremendous speedup was achieved in the process of machine learning modeling and training by using NVIDIA GPU hardware, CUDA 12.2 and cuDNN 8.6 being the libraries for parallel computing and deep learning operations optimization. The strength of Python in scientific computing which is available in the form of NumPy, scikit-learn (sklearn), and pandas was critical in ensuring the numerical, visualization, and machine learning capabilities were in existence. Pydicom was utilized for reading DICOM files and extracting pixel data that were further used in the preprocessing phase. Conda virtual environments has been used to isolate project dependencies, achieved an identical and consistent results. This Python environment made it a quick, easy and agile prototyping and experimentation process due to its user-friendly interface and many machine learning and computer vision libraries available.
\begin{table}[htbp]
  \centering
  \caption{Versions of Python Main Libraries that Were Used}
  \begin{tabular}{|c|c|}
    \hline
    Library & Version \\
    \hline
    numpy & 1.26.3 \\
    scipy & 1.10.1 \\
    pylibjpeg & 2.0.0 \\
    numpy & 1.25.1 \\
    scikit-learn & 1.1.0 \\
    pandas & 2.2.0 \\
    pydicom & 2.1.2 \\
    \hline
  \end{tabular}
\end{table}
\section{Preprocessing}
Our research methodology strategically integrates a critical preprocessing step. This step involves the extraction of key statistical parameters from Digital Imaging and Communications in Medicine (DICOM) images, a fundamental phase aimed at unraveling the intricate characteristics embedded within the images. The overarching goal is to set a robust foundation for the subsequent application of wavelet-based feature extraction techniques.
\subsection{Choice of Statistical Parameters}
The parameter of statistical significance plays a significant role in the approach we take during the breast cancer identification. Our carefully chosen statistical indicators are based on the work of Kumar and Gupta \cite{importancestats}, which is the purpose of designing a medical imaging method that fits the peculiarities of medical imaging.
You can find in our list mean, mode, median, variance, standard deviation, covariance, skewness, and kurtosis among, which are important statistical metrics. Every parameter is selected with a specific goal in mind: to account for different ideas of DICOM picture intensity distributions which are focus of detecting breast cancer.
By an attentive adjustment of our parameters to the information from Kumar and Gupta \cite{importancestats}, we are able to have a guarantee that our methodology is not at random but rather fixed at the particulars of the given breast cancer imaging problems. The logical basis of the selected technique is diversified into its efficiency in detecting abnormalities, outlines, and specific features within breast tissue.
\subsection{Significance of Selected Parameters}
Wavelet and curvelet, which are the multiresolution representations, have been successfully used in image processing applications to zoom up and down on their underlying texture structure \cite{MESELHYELTOUKHY2012123}. The statistical parameters give a rich picture of the fine details of the distributions intensities of pixel within DICOM images especially in the context of breast cancer detection. From the subsequent stanza all essential components are highlighted, with each one outlined in detail to demonstrate its purpose in the manifestation of the appearance of breast tissue.
\begin{itemize}
  \item Mean: The term, “mean” (measure of central tendency) implies the average pixel intensity in a DICOM image. In the aspect of breast cancer detection, the variations in the mean intensity provide vital information. Any abnormalities or deviations from the normal tissue pattern such as presence of irregularities or mass in the breast may be detected through the measurement of mean intensity whose value is different than the expected value. These changes track the tissue variations and help detect small, yet important tissue composition changes. An example of that would be, the mean intensity being elevated in some regions could be an indication of the mass being present which would be additional information for the diagnostics process to use.
  \item Standard Deviation: The standard deviation, a measure of dispersion of which the pixel intensities in a DICOM image reflect the level of variability, characterizes the spread of values in a DICOM image. In the context of mammography and breast cancer determination, standard deviation is very important in showing the pixel values consistency or variation. A larger standard deviation signifies a higher variability in response to frequency and is indicative of a possible microstructure deterioration. A greater standard deviation is likely to mean the existence of identified regions with high fluctuations in intensity level, which could very well signify the presence of abnormalities. Using the mean and standard deviation provides a more complex knowledge of not only the central tendency and variability, but also the strength of the model in showing the subtle patterns specific to breast cancer.
  \item Skewness: Symmetry or Skewness describes which side of the pixel intensity distribution has more pixels. Skewness of data from a normal distribution suggests that there may be some irregularities in breast tissue, which gradually leads to a more thorough subsequent investigation.
  \item Kurtosis: The kurtosis as a statistical feature indicator detects the tail heaviness of the pixel intensity of the DICOM images. When searching for breast cancer on mammograms, the high research values represent outliers or distinct features. The contrast characteristics could help to identify the zones for which the closer inspection may be needed for exclusion of the abnormalities or the masses \cite{li2020diagnostic}. The diagnostic ability of kurtosis in mapping breast cancer cells has been investigated in the studies that used the diffusion kurtosis imaging technique (DKI) as the imaging technique \cite{li2020diagnostic}.
\end{itemize}
\subsection{Computational Implementation}
The computational process involves two main steps: wavelet transformation and statistical parameter computation, described in the paper by Yan et al \cite{YAN2006285} . The wavelet transform, using a particular wavelet type (for example, 'haar') and decomposition levels, produce coefficients. These coefficients are then applied to derive the parameters of interest at various stages of decomposition. Every DICOM file is followed by a text file where the statistical data and wavelet coefficients are stored.

The fact that these statistical parameters are able to encapsulate complex pixel intensity distributions makes them useful foundational information used in wavelet-based feature extraction, thereby allowing for accurate breast cancer detection.
\subsection{Model Development and Validation}
Informed by Barragán-Montero et al.'s comprehensive review on AI in medical imaging \cite{barragan2021artificial}, the breast cancer detection models created and used in this study employed diverse machine learning algorithms: LogisticRegression, RandomForestClassifier, Support Vector Classifier (SVC) and DecisionTreeClassifier.

The development as well the validation of the breast cancer detection models was accomplished by being highly conscious to keep the utilization of the models fully reliable. The datasets had DICOM images of statistical features as its components. The train test split method from the sklearn Modular model selection module was used to to partition the data into training and test sets. This division allowed us to assess predictive power of models on the one hand and objective on the other: it is a crucial issue from the view of practical use.