\chapter{Methodology}
\label{chap3}
\section{Datasets}
This section provides an overview of the diverse datasets employed in this study to train, validate, and test the developed breast cancer detection models. The utilization of multiple datasets enhances the robustness and generalizability of the proposed model, encompassing varied demographics, imaging technologies, and annotation methodologies.
\subsection{VinDr-Mammo Dataset}
The VinDr-Mammo dataset, introduced by Pham et al.\cite{Vindir}, is a pivotal component of this study. This large-scale benchmark dataset for computer-aided detection and diagnosis in full-field digital mammography (FFDM) comprises 5,000 four-view exams. Double-read by experienced mammographers, the dataset provides cancer assessment and breast density following the Breast Imaging Report and Data System (BI-RADS). This contemporary dataset, aligned with current clinical practices, includes detailed annotations of breast abnormalities such as mass, calcification, asymmetries, and architectural distortion, offering comprehensive information for developing and evaluating breast cancer detection algorithms.
\\ Parameters:
\begin{itemize}
  \item Breast Abnormalities: The dataset includes detailed annotations for various abnormalities, including mass, calcification, asymmetries, and architectural distortion.
  \item Clinical Parameters: Patient-related details, such as age, contribute to the dataset's richness and align with real-world clinical scenarios.
  \item BI-RADS Assessment: Each breast is assigned a BI-RADS assessment, providing crucial information for the development and evaluation of breast cancer detection algorithms.
\end{itemize}
This dataset's relevance lies not only in its size but also in the meticulous annotations, making it a valuable resource for developing and validating breast cancer detection models, especially in scenarios where interpretability is of primary concern.
\subsection{RSNA Screening Mammography Breast Cancer Detection Dataset}
The RSNA Screening Mammography Breast Cancer Detection dataset \cite{RSNA}, is integral to the study's exploration of breast cancer identification from screening exams. The dataset contains radiographic breast images of female subjects, and the primary goal is to identify cases of breast cancer in mammograms. With metadata encompassing patient information, imaging details, and cancer-related annotations, this dataset provides a diverse and contemporary collection for training and testing the proposed models, contributing to the dataset's richness, enabling a thorough investigation of breast cancer detection algorithms.
\\ Parameters:
\begin{itemize}
  \item Clinical Details: Metadata includes site ID, patient ID, image ID, laterality (left or right breast), age, implant status, density rating, machine ID, and more.
  \item Cancer Annotations: Annotations related to cancer, biopsy, invasive status, and BIRADS (Breast Imaging Reporting and Data System) assessments contribute to the dataset's utility for cancer detection tasks.
\end{itemize}
This dataset's significance lies in its real-world applicability, aligning with the challenges faced in breast cancer screening programs. The diverse set of parameters provides a holistic view, enabling the study to address nuanced aspects of breast cancer detection.
\section{Computational Environment}
The code implementation for the wavelet-based breast cancer detection system was executed using Python 3.9 as the core programming language. To accelerate machine learning modeling and training, NVIDIA GPU hardware was employed, utilizing CUDA 12.2 and cuDNN 8.6 libraries for parallel computation and optimization of deep learning operations. Key computer vision and image processing functions were facilitated by OpenCV 4.8, including tasks such as loading images, transformations, and visualization. The robust ecosystem of scientific computing tools in Python, such as NumPy, scikit-learn (sklearn), and pandas, played a crucial role in providing essential numerical, visualization, and machine learning capabilities. The pydicom library was employed for reading DICOM files and extracting pixel data, contributing to the preprocessing steps. Conda virtual environments were employed to isolate project dependencies, ensuring consistent and reproducible runs. This Python environment offered efficiency in prototyping and experimentation, given its user-friendly interface and a vast selection of machine learning and computer vision libraries. Table 3.1 details the versions of the main Python libraries used in the experiments.
Library Version
CUDA 12.2
cuDNN 8.6
OpenCV 4.8
numpy 1.25.1
scikit-learn 0.24.2
pandas 1.3.3
pydicom 2.1.2
\section{Preprocessing}
In the dedicated quest for a profound comprehension of the breast cancer detection process, our research methodology strategically integrates a critical preprocessing step. This step involves the extraction of key statistical parameters from Digital Imaging and Communications in Medicine (DICOM) images, a fundamental phase aimed at unraveling the intricate characteristics embedded within the images. The overarching goal is to set a robust foundation for the subsequent application of wavelet-based feature extraction techniques.
\subsection{Choice of Statistical Parameters}
The choice of statistical parameters is crucial in determining our approach when it comes to the identification of breast cancer. Our careful selection of statistical metrics, which we based on the work of Kumar and Gupta \cite{importancestats}, aims to customize the image processing method to the unique characteristics of medical imaging.
Important statistical metrics including mean, mode, median, variance, standard deviation, covariance, skewness, and kurtosis are all included in our list. Every parameter is selected with a specific goal in mind: to capture various aspects of the pixel intensity distributions in DICOM images that are pertinent to the identification of breast cancer.
Through a purposeful adaptation of our parameter selection to the insights offered by Kumar and Gupta \cite{importancestats}, we guarantee that our methodology is not random but rather tailored to the specific issues presented by breast cancer imaging. The rationale behind each chosen measure is grounded in its potential to unveil irregularities, anomalies, and distinctive features within breast tissue.
\subsection{Significance of Selected Parameters}
Multiresolution representations, such as wavelet and curvelet, have proven to be effective in image processing applications, allowing for zooming in and out on the underlying texture structure \cite{MESELHYELTOUKHY2012123}. In the context of breast cancer detection, the chosen statistical parameters offer a comprehensive insight into the nuanced characteristics of pixel intensity distributions within DICOM images. The subsequent lines elaborate on the significance of each parameter, shedding light on their distinct contributions to the representation of breast tissue characteristics.
\begin{itemize}
  \item Mean: The mean, a measure of central tendency, signifies the average pixel intensity within a DICOM image. In the realm of breast cancer detection, variations in mean intensity can serve as crucial indicators. Anomalies or irregularities in breast tissue, such as the presence of masses or abnormalities, may manifest as deviations from the expected mean intensity. Tracking these variations aids in identifying subtle changes in tissue composition. For instance, an elevated mean intensity in a specific region could suggest the presence of a suspicious mass, contributing valuable insights to the diagnostic process.
  \item Standard Deviation: The standard deviation, a measure of dispersion, characterizes the degree of variability in pixel intensities within a DICOM image. In the context of breast cancer detection, standard deviation plays a pivotal role in assessing the consistency or variability of pixel values. Higher standard deviation values indicate greater variability, which may be attributed to irregularities in tissue composition. An increased standard deviation could point to regions with heightened pixel intensity fluctuations, potentially signifying the presence of abnormalities. Combining the mean and standard deviation offers a nuanced understanding of both central tendency and variability, enhancing the model's ability to discern subtle patterns indicative of breast cancer.
  \item Skewness: Skewness quantifies the asymmetry of the pixel intensity distribution. Deviations from a normal distribution, as indicated by skewness, may suggest irregularities in breast tissue, potentially signaling the presence of abnormalities.
  \item Kurtosis: Kurtosis, as a statistical parameter, measures the tail heaviness of the pixel intensity distribution within DICOM images. In the context of breast cancer detection, elevated kurtosis values may serve as indicators of outliers or distinctive features in mammographic images. These distinctive features could potentially highlight regions that require closer examination for the presence of abnormalities or suspicious masses \cite{li2020diagnostic}. The diagnostic potential of kurtosis in characterizing breast tumors has been explored in studies utilizing diffusion kurtosis imaging (DKI) as an imaging technique \cite{li2020diagnostic}.
\end{itemize}
\subsection{Computational Implementation}
The computational process involves two main steps: wavelet transformation and statistical parameter computation, as outlined in the work by Yan et al \cite{YAN2006285}. The wavelet transformation, applied using a specified wavelet type (e.g., 'haar') and decomposition levels, produces coefficients. These coefficients are then utilized to derive the selected statistical parameters at different levels of decomposition. For each DICOM file, a corresponding text file is generated to store the extracted statistical information and wavelet coefficients.

The significance of these statistical parameters lies in their capacity to encapsulate nuanced information about pixel intensity distributions, providing a foundational understanding for subsequent wavelet-based feature extraction in the pursuit of accurate breast cancer detection.

\subsection{Model Development and Validation}
Informed by Barragán-Montero et al.'s comprehensive review on AI in medical imaging \cite{barragan2021artificial}, the breast cancer detection models created and used in this study employed diverse machine learning algorithms: LogisticRegression, RandomForestClassifier, SVC, and DecisionTreeClassifier. This ensemble approach, capturing varied patterns within the data, aimed to enhance robustness and generalizability.

Both the creation and validation of the breast cancer detection models were done with great care to guarantee their dependability. Statistical features derived from DICOM images comprised the datasets. The train test split method from the sklearn model selection module was used to split the datasets into training and testing sets. This stratification made it possible to evaluate model performance objectively, which is essential for practical use.